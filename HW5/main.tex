% Document setup
\documentclass[12pt]{article}
\usepackage[margin=1in]{geometry}
\usepackage{fancyhdr}
\usepackage{lastpage}

\pagestyle{fancy}
\lhead{Richard Whitehill}
\chead{PHYS 804 -- HW \HWnum}
\rhead{\duedate}
\cfoot{\thepage \hspace{1pt} of \pageref{LastPage}}

% Encoding
\usepackage[utf8]{inputenc}
\usepackage[T1]{fontenc}

% Math/Physics Packages
\usepackage{amsmath}
\usepackage{amssymb}
\usepackage{mathtools}
\usepackage{physics}
\usepackage{siunitx}

\AtBeginDocument{\RenewCommandCopy\qty\SI}

% Enumeration/itemize
\usepackage{enumitem}
\newenvironment{parts}
{\begin{enumerate}[label=\textbf{(\alph*)},leftmargin=*,itemsep=-10pt]
}{\end{enumerate}}

% Reference Style
\usepackage{hyperref}
\hypersetup{
    colorlinks=true,
    linkcolor=blue,
    filecolor=magenta,
    urlcolor=cyan,
    citecolor=green
}

\newcommand{\eref}[1]{Eq.~(\ref{eq:#1})}
\newcommand{\erefs}[2]{Eqs.~(\ref{eq:#1})--(\ref{eq:#2})}

\newcommand{\fref}[1]{Fig.~\ref{fig:#1}}
\newcommand{\frefs}[2]{Figs.~\ref{fig:#1}--\ref{fig:#2}}

\newcommand{\tref}[1]{Table~\ref{tab:#1}}
\newcommand{\trefs}[2]{Tables~\ref{tab:#1}-\ref{tab:#2}}

% Figures and Tables 
\usepackage{graphicx}
\usepackage{float}
\usepackage[font=small,labelfont=bf]{caption}

\newcommand{\bef}{\begin{figure}[h!]\begin{center}}
\newcommand{\eef}{\end{center}\end{figure}}

\newcommand{\bet}{\begin{table}[h!]\begin{center}}
\newcommand{\eet}{\end{center}\end{table}}

% tikz
\usepackage{tikz}
\usetikzlibrary{calc}
\usetikzlibrary{decorations.pathmorphing}
\usetikzlibrary{decorations.markings}
\usetikzlibrary{arrows.meta}
\usetikzlibrary{positioning}
\usetikzlibrary{3d}
\usetikzlibrary{shapes.geometric}

% tcolorbox
\usepackage[most]{tcolorbox}
\usepackage{xcolor}
\usepackage{xifthen}
\usepackage{parskip}

\newcommand*{\eqbox}{\tcboxmath[
    enhanced,
    colback=black!10!white,
    colframe=black,
    sharp corners,
    size=fbox,
    boxsep=8pt,
    boxrule=1pt
]}

% problem-solution macros
% \usepackage{adjustbox}
\usepackage{changepage}

\newtcolorbox{probbox}[1][]{
    breakable,
    enhanced,
    boxrule=0pt,
    frame hidden,
    borderline west={4pt}{0pt}{green!50!black},
    colback=green!5,
    before upper=\textbf{Problem #1) \,},
    % \textbf{Problem #1 \ifthenelse{\isempty{#1}}{}{: #1} \\ },
    sharp corners,
    parbox=false
}

% \newtcolorbox{ProblemBox}[1][]{%
%   breakable,
%   enhanced,
%   colback=black!10!white,
%   colframe=black,
%   title={\large #1 \hfill}
% }
\newcommand{\prob}[2]{
\begin{probbox}[#1]
#2
\end{probbox}
}

\newenvironment{solution}{\begin{adjustwidth}{8pt}{8pt}}{\end{adjustwidth}}
\newcommand{\sol}[1]{
\begin{solution}
#1
\end{solution}
}
% \textbf{#1)} #2}

% Miscellaneous Definitions/Settings
\newcommand{\reals}{\mathbb{R}}
\newcommand{\integers}{\mathbb{Z}}
\newcommand{\naturals}{\mathbb{N}}
\newcommand{\rationals}{\mathbb{Q}}
\newcommand{\complexs}{\mathbb{C}}

\setlength{\parskip}{\baselineskip}
\setlength{\parindent}{0pt}
\setlength{\headheight}{14.49998pt}
\addtolength{\topmargin}{-2.49998pt}


\def\HWnum{5}
\def\duedate{November 20, 2024}


\begin{document}

\prob{1}{

Laplace's equation in 2D rectangular coordinates
\begin{align}
    \pdv[2]{f}{x} + \pdv[2]{f}{y} = 0
.\end{align}
The Laplace equation applies to problems in steady state heat conduction, ideal fluid flow, electrostatics, etc.
Write a code for solving 2D Laplace's equation using finite-difference $2^{\rm nd}$-order method and solve 2D Laplace's equation on the square $[0,h] \times [0,h]$ with $h = 10$ and the boundary conditions $f(x,0) = f(x,h) = f(0,y) = 0$ and $f(h,y) = 10$.

\begin{parts}
   
    \item Compute electric potential as a function of $x$ and $y$.

    \item Create a surface plot of the calculated potential.

    \item Run 10, 100, and 1000 iterations, and note when convergence occurs.

    \item Investigate the effect of varying the step size $\Delta h$.
        Draw conclusions regarding the stability and accuracy of the solution for various $\Delta h$.

    \item \textit{Optional}: Investigate the effect of using Gauss-Seidel relaxation versus Jacobi iterations.
        Which converges faster?

\end{parts}

}

\sol{

We can discretize our space and convert this continuous partial differential equation into one relating the value of $f$ at neighboring lattice points using the finite difference method.
Letting $f(x_{i},y_{j}) = f_{i,j}$, we have
\begin{align}
    \frac{f_{i-1,j} - 2 f_{i,j} + f_{i+1,j}}{\Delta x^2} + \frac{f_{i,j-1} - 2 f_{i,j} + f_{i,j+1}}{\Delta y^2} = 0
.\end{align}
Rearranging, we have
\begin{align}
    f_{i,j} = \frac{1}{2} \Big( \frac{1}{\Delta x^2} + \frac{1}{\Delta y^2} \Big)^{-1} \Bigg[ \frac{f_{i-1,j} + f_{i+1,j}}{\Delta x^2} + \frac{f_{i,j-1} + f_{i,j+1}}{\Delta y^2} \Bigg]
.\end{align}
We can devise a method of relaxation based on this equation to construct a sequence for function values at each lattice point such that
\begin{align}
    f_{i,j}^{(n+1)} = \frac{1}{2} \Big( \frac{1}{\Delta x^2} + \frac{1}{\Delta y^2} \Big)^{-1} \Bigg[ \frac{f_{i-1,j}^{(n)} + f_{i+1,j}^{(n)}}{\Delta x^2} + \frac{f_{i,j-1}^{(n)} + f_{i,j+1}^{(n)}}{\Delta y^2} \Bigg]
,\end{align}
where we choose some initial configuration for $f^{(0)}$.

(a)

}


\prob{2}{

Poisson's equation in 2D rectangular coordinates has a nonhomogeneous term
\begin{align}
    \pdv[2]{f}{x} + \pdv[2]{f}{y} = - 4 \pi \rho(x,y)
.\end{align}
Now consider the following boundary conditions for the square region $[0,h] \times [0,h]$ with $h = \pi$,
\begin{align}
\begin{gathered} 
    f(x,0) = \sin{x}, \quad f(x,h) = |\sin{2x}| \\
    f(0,y) = \frac{1}{2} \sin{y}, \quad f(h,y) = 1.2 \sin{y}
,\end{gathered}
\end{align}
and 
\begin{align}
    -4 \pi \rho(x,y) = 
    \begin{cases}
        360 & x = y = h/3 \\
        0 & {\rm otherwise}
    \end{cases}
.\end{align}

\begin{parts}

\item Compute electric potential as a function of $x$ and $y$.

\item Create a surface plot of the calculated potential.

\item Investigate the effect of varying the step size $\Delta h$.
    Draw conclusions regarding the stability and accuracy of the solution for various $\Delta h$.

\end{parts}

}


\sol{

In this problem, we can utilize the same method of relaxation as in problem 1.
We need only add the source term such that
\begin{align} 
    f_{i,j}^{(n+1)} = \frac{1}{2} \Big( \frac{1}{\Delta x^2} + \frac{1}{\Delta y^2} \Big)^{-1} \Bigg[ \frac{f_{i-1,j}^{(n)} + f_{i+1,j}^{(n)}}{\Delta x^2} + \frac{f_{i,j-1}^{(n)} + f_{i,j+1}^{(n)}}{\Delta y^2} - 4 \pi \rho_{i,j} \Bigg]
.\end{align}

}


\prob{3}{

The heat equation for 1D case is
\begin{align}
    \pdv{T(x,t)}{t} = a \pdv[2]{T(x,t)}{x}
.\end{align}
You are given a metal bar of length $L = 1~{\rm m}$ and aligned along the $x$-axis.
It is insulated along its length though not at its ends.
Initially the entire bar is at a uniform temperature of $T_0 = 100^{\circ}{\rm C}$ and then both ends are placed in contact with ice water at $0^{\circ}{\rm C}$.
Heat flows through the non-insulated ends only.

\begin{parts}

\item Write a code for solving $1D$ diffusion equation.

\item Calculate how the temperature varies along the length of the bar as a function of time.

\item Create a 3D plot of $T(x,t)$.

\item Vary the time and space steps in your calculation so that you obtain solutions that are stable in time and vary smoothly in both space and time.

\item Test what happens when the von Neumann stability condition $a \Delta t / \Delta x^2 \leq 0.5$ is not satisfied.

\item \textit{Optional}: The diffusion equation above can be solved analytically as series.
    Compare the analytic and numeric solutions.

\end{parts}

}

\sol{

Following the same logic as the previous two problems, we discretize our space and use the finite difference method to write
\begin{align}
    T_{i,j+1} = T_{i,j} + \frac{a \Delta t}{\Delta x^2} \Big( T_{i-1,j} - 2 T_{i,j} + T_{i+1,j} \Big)
,\end{align}
where $T_{i,j} \equiv T(x_{i},t_{j})$.
With this, we already have a natural time-sequence from this equation given that $T(x,0)$ is supplied as well as the boundary conditions.

}

    
\end{document}
